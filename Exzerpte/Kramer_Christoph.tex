\chapter{Exzerpt Kramer}

Oliver Kramer: Computational Intelligence, Springer 2009.

\section{Vorwort} % 29.11.11 {{{1

VII: Traditionelle Methoden der Informatik beruhen auf der
symbolischen Repräsentation und statischen Zustände,
besonders zur Modellierung natürlicher Phänomene haben sie
ihre Grenzen.

Bedürfnis nach flexibleren, dynamischeren Methoden mit
höherer Fehlertoleranz. Dem versucht die \CI{} gerecht zu
werden.

\CI{} umfasst Methoden zur
    intelligenten Informationsverarbeitung
    zur Optimierung
    zur Steuerung und Regelung
    zur Klassifikation

Rein klassische Themen der \CI{} sind:
    evolutionäre Algorithmen
    Fuzzy-Systeme
    neuronale Netze
neuere Ansätze:
    Schwarmintelligenz
    künstliche Immunsysteme
    Reinforcement Learning

\section{Überblick \CI{}} % 29.11.11 {{{1

1: \CI{} begann in den Fünfziger Jahren des letzten
Jahrhunderts mit den ersten neuronalen Netzwerken von
Rosenblatt. Sie sind der Beginn der evolutionären
Algorithmen und der Fuzzy-Logik.

2: Einige Techniken der \CI{} sind eng verwandt mit denen
der \AI{}. 

Klassifikation Verfahren: Anwendungsbereich
    Neuronale Netze: Klassifikation
    Reinforcement Learning: Verhaltenssteuerung
    Fuzzy-Logik: Steuerung, Clustern
    Künstliche Immunsysteme: Mustererkennung, Optimierung
    \SI{}: Optimierung, Emergenz
    Evolutionäre Algorithmen: Optimierung, Selbstadaption.

Seit ihren Anfängen bei John von Neumann und Alan Turing
wird Informationsverarbeitung als \enquote{intelligent}
bezeichnet, 
\textquote[{\cite[2]{bib:Kramer_2009_intelligence}}]
{wenn die Algorithmen menschenähnliche Leistungen zu
vollbringen in der Lage sind}. Dazu zählt man Lernfähigkeit,
Fähigkeit zur Anpassung an sich verändernde Umstände.

3: Weiter charakterisiert man \II{} über die Aufgaben, für
die sie eingesetzt wird:
    Suche und Optimierung
    Klassifikation und Gruppierung
    Erkennung von Mustern
    Steuerung von Verhalten und komplexe Regelung

Schwierig ist heute nach
\citeauthor{bib:Kramer_2009_intelligence}
besonders die indeutige Begriffsabgrenzung
zwischen \AI{}, \CI{}, measchinellem
Lernen, Bionik, Soft Computing oder Natural Computung und
einigen anderen macht heute Probleme.

Algorithmen zur intelligenten Informationsverarbeitung
gehören zu den beiden Hauptgebieten \CI{} und \AI{}.

Schwache \AI{}: Ein Algorithmus vollbringt zur Problemlösung
menschenähnliche Leistungen.
Starkte \AI{}: menschliche Kognition soll nachgebaut werden.

4: Traditionelle Ansätze der \AI{} beruhen auf statischen
und diskreten Suchverfahren. Dies reicht für viele
natürliche Phänomene nicht, die \CI{} will mit ihren
subsymbolischen Techniken dem gerecht werden.

5: Damit bietet die \CI{} folgenden Vorteile:
    Fehlertoleranz
    Parallelität
    Einfachheit der Modellierung
    Effiziente Näherung

\CI{} lehnt sich eng an natürliche Problemlösungsprozesse
an. Dazu wird ein passendes biologisches Modell gesucht, das
in eine passende Rechenvorschrift übersetzt wird. Der Weg
führt von der Inspiration in der Biologie, über die
theoretische Modellierung hin zur anwendungsspezifischen
Anpassung.

Im Gegensatz zur \CI{} ist die Bionik hauptsächlich an
biologischen Strukturen interessiert und kaum an
algorithmischen Konzepten.

8: \SI{} dient zur Lösung von Approximation von Lösungen für
Optimierungsprobleme, bei denen kaum Wissen zur Verfügung
steht. Die Lösungsqualität wird über Pheromone auf die
einzelnen Komponenten der Lösung verteilt.

Die Verfahren der \SI{} ähneln der der evolutionären
Algorithmen sehr. Beide Techniken sind populationsbasiert
und verwenden stochastische Operatoren zur Variation ihrer
Lösungskandidaten.
