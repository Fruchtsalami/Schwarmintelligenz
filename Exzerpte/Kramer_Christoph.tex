\chapter{Exzerpt Kramer}

Oliver Kramer: Computational Intelligence, Springer 2009.

\section{Vorwort} % 29.11.11 {{{1

VII: Traditionelle Methoden der Informatik beruhen auf der
symbolischen Repräsentation und statischen Zuständen,
besonders zur Modellierung natürlicher Phänomene haben sie
ihre Grenzen.

Bedürfnis nach flexibleren, dynamischeren Methoden mit
höherer Fehlertoleranz. Dem versucht die \CI gerecht zu
werden.

\CI umfasst Methoden zur
    intelligenten Informationsverarbeitung
    zur Optimierung
    zur Steuerung und Regelung
    zur Klassifikation

Rein klassische Themen der \CI sind:
    evolutionäre Algorithmen
    Fuzzy-Systeme
    neuronale Netze

neuere Ansätze:
    \SI
    künstliche Immunsysteme
    Reinforcement Learning

\section{Überblick \CI} % 29.11.11 {{{1

1: \CI begann in den Fünfziger Jahren des letzten
Jahrhunderts mit den ersten neuronalen Netzwerken von
Rosenblatt. Sie sind der Beginn der evolutionären
Algorithmen und der Fuzzy-Logik.

2: Einige Techniken der \CI sind eng verwandt mit denen
der \AI. 

Klassifikation Verfahren: Anwendungsbereich
    Neuronale Netze: Klassifikation
    Reinforcement Learning: Verhaltenssteuerung
    Fuzzy-Logik: Steuerung, Clustern
    Künstliche Immunsysteme: Mustererkennung, Optimierung
    \SI: Optimierung, Emergenz
    Evolutionäre Algorithmen: Optimierung, Selbstadaption.

Seit ihren Anfängen bei John von Neumann und Alan Turing
wird Informationsverarbeitung als \enquote{intelligent}
bezeichnet, 
\textquote[{\cite[2]{bib:Kramer_2009_intelligence}}]
{wenn die Algorithmen menschenähnliche Leistungen zu
vollbringen in der Lage sind}. Dazu zählt man Lernfähigkeit,
Fähigkeit zur Anpassung an sich verändernde Umstände.

3: Weiter charakterisiert man \II über die Aufgaben, für
die sie eingesetzt wird:
    Suche und Optimierung
    Klassifikation und Gruppierung
    Erkennung von Mustern
    Steuerung von Verhalten und komplexe Regelung

Schwierig ist heute nach
\citeauthor{bib:Kramer_2009_intelligence}
besonders die eindeutige Begriffsabgrenzung
zwischen \AI, \CI, maschinellem
Lernen, Bionik, Soft Computing oder Natural Computing und
einigen anderen macht heute Probleme.

Algorithmen zur intelligenten Informationsverarbeitung
gehören zu den beiden Hauptgebieten \CI und \AI.

Schwache \AI: Ein Algorithmus vollbringt zur Problemlösung
menschenähnliche Leistungen.
Starke \AI: menschliche Kognition soll nachgebaut werden.

4: Traditionelle Ansätze der \AI beruhen auf statischen
und diskreten Suchverfahren. Dies reicht für viele
natürliche Phänomene nicht, die \CI will mit ihren
subsymbolischen Techniken dem gerecht werden.

5: Damit bietet die \CI folgenden Vorteile:
    Fehlertoleranz
    Parallelität
    Einfachheit der Modellierung
    Effiziente Näherung

\CI lehnt sich eng an natürliche Problemlösungsprozesse
an. Dazu wird ein passendes biologisches Modell gesucht, das
in eine passende Rechenvorschrift übersetzt wird. Der Weg
führt von der Inspiration in der Biologie, über die
theoretische Modellierung hin zur anwendungsspezifischen
Anpassung.

Im Gegensatz zur \CI ist die Bionik hauptsächlich an
biologischen Strukturen interessiert und kaum an
algorithmischen Konzepten.

8: \SI dient zur Lösung von Approximation von Lösungen für
Optimierungsprobleme, bei denen kaum Wissen zur Verfügung
steht. Die Lösungsqualität wird über Pheromone auf die
einzelnen Komponenten der Lösung verteilt.

Die Verfahren der \SI ähneln der der evolutionären
Algorithmen sehr. Beide Techniken sind populationsbasiert
und verwenden stochastische Operatoren zur Variation ihrer
Lösungskandidaten.

\section{\SI} % {{{1

Unter \enquote{\SI} versteht \citeauthor{bib:Kramer_2009_intelligence}
erfolgreiche Systeme, die in der Natur existieren, die, obwohl als einzelne eher
einfach strukturiert, sich zu Gruppen zusammenschließen können, um gemein
zielgerichtet zu handeln. So würden sie den Nachteil der einfachen Fähigkeiten
ihrer mit ihrer Masse und \textquote[{\cite[41]{bib:Kramer_2009_intelligence}}]
{die dadurch erreichte massive Parallelität} ausgleichen. Auf
\enquote{emergente} Weise werden so zielgerichtetes Handeln ermöglicht.

\citeauthor{bib:Kramer_2009_intelligence} sieht das Problem, daß sich nicht
allgemein beantworten läßt, ob eine große Zahl \enquote{einfacher Einheiten}
besser Probleme lösen kann als eine kleine Anzahl komplexer Systeme.
Dennoch zeige sich im beobachteten Verhalten erfolgreiche Problemlösekompetenz.
Als Beispiel für einfache Einheiten, die als Schwarm Problemlösekompetenz
entfalten, nennt er das Beispiel der Ameisen.

42: \ST und Emergenz sind die beiden zentralen Konzepte der \SI.
Unter \ST versteht der Autor, daß Individuen eines Schwarms über ihre Umwelt
miteinander kommunizieren. Bei Ameisen geschehe diese über Pheromone, die von
anderen Ameisen wahrgenommen werden.
Emergenz soll bezeichnen, daß Individuen eines Schwarms aufgrund des
Zusammenspiels ihrer Fähigkeiten \enquote{intelligentes} Verhalten
hervorbringen, wobei die das Gesamtsysteme höhere Eigenschaften und Fähigkeiten
entfalte als seine Mitglieder.

43: Weitere Beispiele für Schwarmverhalten liefern Vogel, Insekten- oder
Fischschwärme. In ihnen spiele das Individuum nur eine untergeordnete Rolle, da
es seine Bewegungen nach einfachen Regeln anpaßt, die hauptsächlich vom
unmittelbaren Nachbarn abhängen würden. Daraus entstünden Bewegungsmuster, die
sich bspw. bei der Pfadplanung einsetzen ließen.

Darüber hinaus ließe sich \textquote[{\cite[43]{bib:Kramer_2009_intelligence}}]
{mit Schwarmbildung die emergente Entwicklung der
Verhaltenssteuerung einer großen Menge von Individuen simulieren, die sich
hauptsächlich an ihrer Umgebung orientieren.} Beispiele dafür seien ökonomische
Einheiten bei der Marktanalyse aber auch die Entwicklung kultureller
Informationen.

Als erster simulierte Craig Reynolds 1987 erfolgreich Schwarmbildung. Dabei habe
er drei maßgebliche Prinzipien identifiziert:
- Zusammenhalt: jeder Partikel orientiere sich an der Position seiner Nachbarn
- Ausrichtung: jeder Partikel bewege sich ähnlich der Bewegungsrichtung seiner
    Nachbarn
- Trennung: Jeder Partikel vermeidet Kollisionen mit seinen Nachbarn.

Mit diesen Regeln sei bereits eine realistische Simulation von Schwarmbewegungen
möglich.

44: Diskussion des Verfahrens.

45: Beispiel der Simulation künstlichen Lebens

46: In der rechnergestützten Simulation von Schwarmbewegungen zeigten sich
Muster, wie sie sich auch bei lebenden Organismen zeigen. Deshalb sieht
\citeauthor{bib:Kramer_2009_intelligence} interessante Anknüpfungspunkte zu \AI,
Sozionik und Biologie.

% Optimierung aus reinem Bewegungsverhalten von Schwärmen

46: Kennedy und Eberhart(?) haben 1995 einen Algorithmus in Anlehnung an natürliches
Schwarmverhalten vorgeschlagen, der darum als \PSO bezeichnet wird.

47: Kontinuierliche Variante

50: Diskrete Variante

% Optimierung über Informationsaustausch von Schwärmen

51: Der \AAL modelliert das Verhalten von Ameisen zu Optimierungszwecken. Die
Ameisenheuristik wurde 1992 erstmals von Dorigo vorgeschlagen. Nach
\citeauthor{bib:Kramer_2009_intelligence} bilden Ameisen in der Natur komplexe
Schwärme, die arbeitsteilig Aufgaben wie Nestbau, Brutpflege und Nahrungssuche
übernehmen. Die Individuen kommunizieren dabei über Pheromone miteinander.

52: Double-Bridge-Experiment: Ameisen auf Futtersuche versuche verschiedene Wege
und hinterlassen dabei eine Pheromonspur. Auf der kürzeren der Wege wird sich
bald das Pheromon dichter verteilen, was die folgenden Ameisen als Indiz für
einen besseren Weg nehmen, und so weiter die Spur intensivieren.

53: Details des \AAL.
Erste Stufe des Algorithmus: Stochastische Auswahl des Pfades unter Einbezug
weiterer bekannter Größen.

54: Details des \AAL.
Zweite Stufe des Algorithmus: Pheromonablage.

55: Verwitterungsfaktor.

Rest des Kapitels: Beispielrechnung Städtereise.
