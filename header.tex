%----------------------------------------------------------------------------
%  Koma-Skript Optionen
%---------------------------------------------------------------------------
\documentclass[
	a4paper,
	12pt,
	%twocolumn,
	headings=normal     % big,small
]{scrbook} %scrartcl,scrbook,scrreprt

%---------------------------------------------------------------------------
%  Spracheinstellungen
%---------------------------------------------------------------------------

\usepackage[utf8]{inputenc}
\usepackage[T1]{fontenc}
\usepackage[english,ngerman]{babel}
%Trennungsregeln mit
\hyphenation{Sil-ben-trenn-ung Hun-de-ku-chen}

%---------------------------------------------------------------------------
%  Kopf- und Fußzeile
%---------------------------------------------------------------------------

% Gestaltung von Kopf- und Fußzeile über koma-Script
\usepackage[automark]{scrpage2}
	\ifoot[]{}
	\cfoot[]{}
	\ofoot[]{}
	\automark[section]{section}
	\chead[]{\headmark}
	\ohead[]{}
	\ohead[]{\pagemark}
	\pagestyle{scrheadings}

%---------------------------------------------------------------------------
%  Seitenlayout
%---------------------------------------------------------------------------

\usepackage{geometry}
\geometry{%
	a4paper,
	top=3.5cm,
	left=2.3cm,
	right=4.5cm,
	bottom=4.5cm,
}

%---------------------------------------------------------------------------
%  Schriften
%---------------------------------------------------------------------------

\usepackage{mathptmx}
\usepackage[scaled]{helvet}

%---------------------------------------------------------------------------
%  Bibliographie
%---------------------------------------------------------------------------

\usepackage[style=authoryear,
            hyperref=true,
            isbn=false,
            firstinits=true,
]{biblatex}
\bibliography{Data.bib}       % Gibt den Ort der Bibliographiedatei an
\usepackage[babel]{csquotes}

%---------------------------------------------------------------------------
%  Graphiken
%---------------------------------------------------------------------------

\usepackage{graphicx}

%---------------------------------------------------------------------------
%  Tabellen
%---------------------------------------------------------------------------

\usepackage{booktabs}	% für schöne Tabellen

%---------------------------------------------------------------------------
%  Seiten- und Absatzumbruch
%---------------------------------------------------------------------------

\clubpenalty10000   % keine Schusterjungen
\widowpenalty10000  % keine Hurenkinder
\parindent0pt	% kein Einzug
\flushbottom	% erzwinge gleich volle Seiten

%---------------------------------------------------------------------------
%  Für wissenschaftliche Arbeiten
%---------------------------------------------------------------------------

\newcommand{\fullname}{Vorname Nachname}
\newcommand{\email}{vorname.nachname@uni-ulm.de}
\newcommand{\titel}{Titel der Arbeit}
\newcommand{\jahr}{2009}
\newcommand{\matnr}{123456}
\newcommand{\gutachterA}{Prof.\ Dr.\ Streng Geheim}
\newcommand{\gutachterB}{Prof.\ Dr.\ Un Leserlich}
\newcommand{\betreuer}{Betreuername}
\newcommand{\fakultaet}{Ingenieurwissenschaften\\und Informatik}
%\newcommand{\fakultaet}{Mathematik und\\Wirtschaftswissenschaften}
%\newcommand{\fakultaet}{Naturwissenschaften}
%\newcommand{\fakultaet}{Medizin}
% nun noch unten das Institut einsetzen
\newcommand{\institut}{Institut für Irgendetwas}

\usepackage{setspace}	% \onehalfspacing, \singlespacing, \doublespacing

%---------------------------------------------------------------------------
%  Abkürzungen
%---------------------------------------------------------------------------

\usepackage{xspace}

\newcommand{\CI}{Computational Intelligence\xspace}
\newcommand{\AI}{Artificial Intelligence\xspace}
\newcommand{\SI}{Swarmintelligence\xspace}
\newcommand{\II}{Intelligente Informationsverarbeitung\xspace}


%---------------------------------------------------------------------------
%  Pdf-Setup
%---------------------------------------------------------------------------

\usepackage{microtype}	    % verbesserter Textsatz

\usepackage[]{hyperref} 
\hypersetup{
    pdftitle=\titel{},
    pdfauthor=\fullname{},
    pdfsubject={Abschlussarbeit},
    colorlinks=false,	% farbige Links abschalten
    pdfborder=0 0 0	% keine Box um die Links!
}
